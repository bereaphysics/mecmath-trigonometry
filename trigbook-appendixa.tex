\addchap[Appendix A:$\quad$Answers and Hints to Selected Exercises]{Appendix A}
\textsf{\textbf{\Large Answers and Hints to Selected Exercises}}
\begin{multicols}{2}
\section*{Chapter 1}
\subsection*{Section 1.1 (p. \pageref{sec1dot1})}
\textbf{1.} $115\Degrees$ \quad \textbf{3.} $A=52\Degrees$, $B=104\Degrees$ \quad
\textbf{5.} $45\Degrees$\\\textbf{7.} $A=9\Degrees$, $B=81\Degrees$ \quad
\textbf{8.} $0.011\Degrees$ and $89.989\Degrees$\\\textbf{9.} $25$ miles \quad \textbf{10.}
$111.8$ ft\\\textbf{15.} Hint: Are the opposite sides of the four-sided figure inside the circle
parallel?
\subsection*{Section 1.2 (p. \pageref{sec1dot2})}
\textbf{1.} $\sin\;A = 5/13$, $\cos\;A = 12/13$, $\tan\;A = 5/12$,\\
$\csc\;A = 13/5$, $\sec\;A = 13/12$, $\cot\;A = 12/5$;\\
$\sin\;B = 12/13$, $\cos\;B = 5/13$, $\tan\;B = 12/5$,\\
$\csc\;B = 13/12$, $\sec\;B = 13/5$, $\cot\;B = 5/12$\\
\textbf{3.} $\sin\;A = 7/25$, $\cos\;A = 24/25$, $\tan\;A = 7/24$,\\
$\csc\;A = 25/7$, $\sec\;A = 25/24$, $\cot\;A = 24/7$;\\
$\sin\;B = 24/25$, $\cos\;B = 7/25$, $\tan\;B = 24/7$,\\
$\csc\;B = 25/24$, $\sec\;B = 25/7$, $\cot\;B = 7/24$\\
\textbf{5.} $\sin\;A = 9/41$, $\cos\;A = 40/41$, $\tan\;A = 9/40$,\\
$\csc\;A = 41/9$, $\sec\;A = 41/40$, $\cot\;A = 40/9$;\\
$\sin\;B = 40/41$, $\cos\;B = 9/41$, $\tan\;B = 40/9$,\\
$\csc\;B = 41/40$, $\sec\;B = 41/9$, $\cot\;B = 9/40$\\
\textbf{7.} $\sin\;A = 1/\sqrt{10}$, $\cos\;A = 3/\sqrt{10}$, $\tan\;A = 1/3$,\\
$\csc\;A = \sqrt{10}$, $\sec\;A = \sqrt{10}/3$, $\cot\;A = 3$;\\
$\sin\;B = 3/\sqrt{10}$, $\cos\;B = 1/\sqrt{10}$, $\tan\;B = 3$,\\
$\csc\;B = \sqrt{10}/3$, $\sec\;B = \sqrt{10}$, $\cot\;B = 1/3$\\
\textbf{9.} $\sin\;A = 5/6$, $\cos\;A = \sqrt{11}/6$, $\tan\;A = 5/\sqrt{11}$,\\
$\csc\;A = 6/5$, $\sec\;A = 6/\sqrt{11}$, $\cot\;A = \sqrt{11}/5$;\\
$\sin\;B = \sqrt{11}/6$, $\cos\;B = 5/6$, $\tan\;B = \sqrt{11}/5$,\\
$\csc\;B = 6/\sqrt{11}$, $\sec\;B = 6/5$, $\cot\;B = 5/\sqrt{11}$\\
\textbf{11.} $\cos\;A = \sqrt{7}/4$, $\tan\;A = 3/\sqrt{7}$,
$\csc\;A = 4/3$, $\sec\;A = 4/\sqrt{7}$, $\cot\;A = \sqrt{7}/3$\\
\textbf{13.} $\sin\;A = \sqrt{6}/\sqrt{10}$, $\tan\;A = \sqrt{6}/2$,\\
$\csc\;A = \sqrt{10}/\sqrt{6}$, $\sec\;A = \sqrt{10}/2$, $\cot\;A = 2/\sqrt{6}$\\
\textbf{15.} $\sin\;A = 5/\sqrt{106}$, $\cos\;A = 9/\sqrt{106}$,\\
$\csc\;A = \sqrt{106}/5$, $\sec\;A = \sqrt{106}/9$, $\cot\;A = 9/5$\\
\textbf{17.} $\sin\;A = \sqrt{40}/7$, $\cos\;A = 3/7$,\\$\tan\;A = \sqrt{40}/3$,
$\csc\;A = 7/\sqrt{40}$, $\cot\;A = 3/\sqrt{40}$\\
\textbf{19.} $\cos\;3\Degrees$ \quad \textbf{21.} $\sin\;44\Degrees$ \quad
\textbf{23.} $\csc\;13\Degrees$\\
\textbf{25.} $\sin\;77\Degrees$ \quad
\textbf{27.} $\tan\;80\Degrees$ \quad \textbf{30.} Hint: Draw a right triangle with an acute
angle $A$.\\
\textbf{33.} Hint: Draw two right triangles whose hypotenuses are the same length.\\
\textbf{37.} \textbf{(a)} $\sqrt{13}/4$ \textbf{(b)} $4\sqrt{3}/\sqrt{13}$
\textbf{(c)} $3/\sqrt{13}$
\subsection*{Section 1.3 (p. \pageref{sec1dot3})}
\textbf{1.} $102.7$ ft \quad \textbf{3.} $241.1$ ft \quad \textbf{4.} $274$ ft \quad
\textbf{7.} $1062$ mi \quad \textbf{9.} $0.476$ in \quad \textbf{11.} $1.955$ in \quad
\textbf{13.} $0.4866$ in \quad \textbf{14.} Partial answer: $DE=a\;\cot\;\theta\;\,\cos^2\,\theta$
\quad \textbf{15.} $c=13$, $A=22.6\Degrees$, $B=67.4\Degrees$ \quad \textbf{17.} $a=0.28$, $c=2.02$,
$B=82\Degrees$ \quad \textbf{19.} $b=6.15$, $c=6.84$, $B=64\Degrees$\\
\textbf{21.} $a=6.15$, $c=6.84$, $A=64\Degrees$ \quad \textbf{23.} $a=\sqrt{2}$, $b=\sqrt{2}$,
$B=45\Degrees$ \quad \textbf{25.} \textbf{(a)} $0.944$ cm\\\textbf{(b)} $2.112$ cm \quad
\textbf{27.} \textbf{(a)} $\sqrt{3}\;a$ \quad \textbf{(b)} $35.26\Degrees$\\\textbf{29.} $1379.5$ ft $= 0.2613$ mi
\subsection*{Section 1.4 (p. \pageref{sec1dot4})}
\textbf{1.} QII \quad \textbf{3.} QIV \quad \textbf{5.} negative $y$-axis\\\textbf{7.} QIII
\quad \textbf{9.} QIV \quad \textbf{11.} QI, QIII \quad \textbf{13.} QI, QIV \quad
\textbf{15.} QI, QII \quad \textbf{17.} $43\Degrees$ \quad \textbf{19.} $54\Degrees$ \quad
\textbf{21.} $85\Degrees$ \quad \textbf{23.} $\sin\;\theta = \sqrt{3}/2$ and $\tan\;\theta =
-\sqrt{3}$; $\sin\;\theta = -\sqrt{3}/2$ and $\tan\;\theta = \sqrt{3}$\\\textbf{25.}
$\sin\;\theta = \sqrt{21}/5$ and $\tan\;\theta = \sqrt{21}/2$;\\$\sin\;\theta = -\sqrt{21}/5$ and
$\tan\;\theta = -\sqrt{21}/2$\\\textbf{27.} $\cos\;\theta = \sqrt{3}/2$ and $\tan\;\theta =
1/\sqrt{3}$;\\$\cos\;\theta = -\sqrt{3}/2$ and $\tan\;\theta = -1/\sqrt{3}$\\
\textbf{29.} $\cos\;\theta = \pm 1$ and $\tan\;\theta = 0$\\\textbf{31.} $\cos\;\theta = 0$
and $\tan\;\theta$ is undefined\\\textbf{33.} $\sin\;\theta = 1/\sqrt{5}$ and
$\cos\;\theta = -2/\sqrt{5}$;\\$\sin\;\theta = -1/\sqrt{5}$ and $\cos\;\theta =
2/\sqrt{5}$\\\textbf{35.} $\sin\;\theta = 5/13$ and
$\cos\;\theta = 12/13$;\\$\sin\;\theta = -5/13$ and $\cos\;\theta =
-12/13$ \quad \textbf{37.} No \quad \textbf{39.} No
\subsection*{Section 1.5 (p. \pageref{sec1dot5})}
\textbf{1.} \textbf{(a)} $328\Degrees$ \textbf{(b)} $148\Degrees$ \textbf{(c)} $212\Degrees$ \quad
\textbf{3.} \textbf{(a)} $248\Degrees$ \textbf{(b)} $68\Degrees$ \textbf{(c)} $292\Degrees$ \quad
\textbf{7.} $25\Degrees$, $155\Degrees$ \quad \textbf{9.} $65\Degrees$, $295\Degrees$ \quad
\textbf{11.} $38\Degrees$, $218\Degrees$ \quad \textbf{13.} $169\Degrees$,
$191\Degrees$\\\textbf{15.} $D=\left( \frac{ab^2}{a^2 + b^2}, \frac{a^2 b}{a^2 + b^2} \right)$
\section*{Chapter 2}
\subsection*{Section 2.1 (p. \pageref{sec2dot1})}
\textbf{1.} $b = 7.4$, $c = 15.1$, $C = 120\Degrees$ \quad \textbf{3.} $a = 9.7$, $b = 10.7$, $C =
95\Degrees$ \quad \textbf{5.} $b = 65.1$, $B = 136.5\Degrees$, $C = 18.5\Degrees$ \quad
\textbf{7.} No solution \quad \textbf{9.} $b = 24.9$, $B = 59.9\Degrees$, $C = 70.1\Degrees$;
$b = 9.9$, $B = 20.1\Degrees$, $C = 109.9\Degrees$ \quad \textbf{11.} $422$ mi/hr \quad \textbf{15.}
$5.66$ cm and $12.86$ cm \quad \textbf{16.} Hint: Think geometrically.
\subsection*{Section 2.2 (p. \pageref{sec2dot2})}
\textbf{1.} $a = 10.6$, $B = 40.9\Degrees$, $C = 79.1$ \quad \textbf{3.} $A = 47.9\Degrees$,
$b = 8.2$, $C = 72.1\Degrees$ \quad \textbf{5.} No solution \quad \textbf{7.} $4.13$ and $8.91$ cm
\quad \textbf{9.} $50.5\Degrees$, $59\Degrees$, $70.5\Degrees$\\\textbf{11.} $7$ cm \quad
\textbf{15.} Hints: One of the angles in the formulas is a right angle; also, use the definition of
cosine.
\subsection*{Section 2.3 (p. \pageref{sec2dot3})}
\textbf{1.} $A = 79.1\Degrees$, $B = 40.9\Degrees$, $c = 10.6$ \quad \textbf{3.} $A = 47.9\Degrees$,
$b = 8.2$, $C = 72.1\Degrees$ \quad \textbf{5.} No \quad \textbf{6.} Yes \quad \textbf{11.} Hint:
Think of Exercise 10.
\subsection*{Section 2.4 (p. \pageref{sec2dot4})}
\textbf{1.} $22.55$ \quad \textbf{3.} $9.21$ \quad \textbf{5.} $\frac{3}{4}\sqrt{15}
\approx 2.905$\\\textbf{7.} $12.21$ \quad \textbf{9.} Hints: The diagonals break the quadrilateral
into four triangles; also, consider formulas (\ref{eqn:areacase1a})-(\ref{eqn:areacase1c}).
\subsection*{Section 2.5 (p. \pageref{sec2dot5})}
\textbf{1.} $R = 2.63$, $r = 0.69$ \quad \textbf{3.} $R = 3.51$, $r = 1.36$ \quad
\textbf{5.} $R = 24.18$, $r = 1.12$ \quad \textbf{12.} \textbf{(c)} Twice as large
\textbf{(d)} Hint: Bisect each angle.
\section*{Chapter 3}
\subsection*{Section 3.1 (p. \pageref{sec3dot1})}
\textbf{1.} $\theta = 270\Degrees$ \quad \textbf{3.} Hint: See Example \ref{exmp:elimtheta}. \quad
\textbf{19.} $\tan\;\theta = \pm\,\sin\;\theta / \sqrt{1 - \sin^2 \;\theta} =
\pm\,\sqrt{1 - \cos^2 \;\theta} / \cos\;\theta$
\subsection*{Section 3.2 (p. \pageref{sec3dot2})}
\textbf{3.} $\sin\;(A+B) = \frac{1020}{1189}$, $\cos\;(A+B) = -\frac{611}{1189}$,
$\tan\;(A+B) = -\frac{1020}{611}$ \quad \textbf{4.} $(\sqrt{6} + \sqrt{2})/4$\\
\textbf{5.} $2 - \sqrt{3}$ \quad \textbf{15.} Hint: For $a \ne 0$ and $b \ne 0$, draw a right
triangle with legs of lengths $a$ and $b$.
\subsection*{Section 3.3 (p. \pageref{sec3dot3})}
\textbf{9.} Hint: Is $\sin\;A + \cos\;A$ always positive? \textbf{11.} $1/2$
\subsection*{Section 3.4 (p. \pageref{sec3dot4})}
\textbf{13.} Hint: One way to do this is with the Law of Tangents. Another way is with the Law of
Sines.
\section*{Chapter 4}
\subsection*{Section 4.1 (p. \pageref{sec4dot1})}
\textbf{1.} $\pi/45$ \quad \textbf{3.} $13\pi/18$ \quad \textbf{5.} $-3\pi/5$ \quad
\textbf{7.} $36\Degrees$ \quad \textbf{9.} $174\Degrees$
\subsection*{Section 4.2 (p. \pageref{sec4dot2})}
\textbf{1.} $9.6$ cm \quad \textbf{3.} $11\pi$ in \quad \textbf{5.} $54.94$ in\\
\textbf{7.} $12.86$ ft \quad \textbf{8.} $34.18$ \quad \textbf{9.} $38.26$\\
\textbf{11.} $3.392$ and $9.174$ \quad \textbf{12.} $3.105828541$
\subsection*{Section 4.3 (p. \pageref{sec4dot3})}
\textbf{1.} $1.512~\text{cm}^2$ \quad \textbf{3.} $24.5~\text{m}^2$ \quad
\textbf{5.} $269.1~\text{cm}^2$ \quad \textbf{7.} $5~\text{cm}^2$ \quad
\textbf{9.} $\pi/2~\text{cm}^2$ \quad \textbf{11.} $0.017~\text{cm}^2$ \quad
\textbf{13.} $21.46$ \quad \textbf{15.} $48.17$ \quad \textbf{17.} $0.522~\text{m}^2$\\
\textbf{19.} Sector area is quadrupled, arc length is doubled.
\subsection*{Section 4.4 (p. \pageref{sec4dot4})}
\textbf{1.} $\nu=6$ m/sec, $\omega=1.5$ rad/sec\\
\textbf{3.} $\nu=6.6$ m/sec, $\omega=0.94$ rad/sec\\
\textbf{5.} $\nu=3.75$ m/sec, $\omega=1.875$ rad/sec\\
\textbf{7.} $3.375$ rad \quad \textbf{9.} $32$ rpm and $21.33$ rpm\\
\textbf{11.} $40.84$ in/sec
\section*{Chapter 5}
\subsection*{Section 5.1 (p. \pageref{sec5dot1})}
\textbf{13.} Partial answer: $\sec\;\theta = OQ$
\subsection*{Section 5.2 (p. \pageref{sec5dot2})}
\textbf{1.} amplitude $= 3$, period $= 2$, phase shift = $0$ \quad
\textbf{3.} amplitude $= 1$, period $= 2\pi/5$, phase shift = $-3/5$ \quad
\textbf{5.} amplitude $= 1$, period $= 2\pi/5$, phase shift = $-\pi/5$ \quad
\textbf{7.} amplitude $= 1$, period $= \pi$, phase shift = $3\pi/2$\\
\textbf{9.} amplitude undefined, period $= \pi/2$, phase shift = $3\pi/2$ \quad
\textbf{11.} amplitude undefined, period $= \pi$, phase shift = $1/2$\\
\textbf{13.} max. at $x=\pm\,\sqrt{\pi/2}$, $\pm\,\sqrt{5\pi/2}$, $\pm\,\sqrt{9\pi/2}$, $...$\\
min. at $x=\pm\,\sqrt{3\pi/2}$, $\pm\,\sqrt{7\pi/2}$, $\pm\,\sqrt{11\pi/2}$, $...$\\
\textbf{15.} amplitude $= 0.5$, period $= \pi$ \quad \textbf{17.} out of phase \quad
\textbf{18.} in phase \quad \textbf{19.} amplitude $= \sqrt{34}$, period $= 2$ \quad
\textbf{21.} amplitude $= 2\,\sqrt{2}$, period $= 2\pi$ \quad
\textbf{23.} $2\pi$ \quad \textbf{25.} $6$ \quad \textbf{27.} amplitude envelope: $y=\pm\,x^2$ \quad
\textbf{29.} No
\subsection*{Section 5.3 (p. \pageref{sec5dot3})}
\textbf{1.} $\pi/4$ \quad \textbf{3.} $0$ \quad \textbf{5.} $\pi$ \quad \textbf{7.} $\pi/2$ \quad
\textbf{9.} $0$\\\textbf{11.} $-\pi/3$ \quad \textbf{13.} $\pi/7$ \quad \textbf{15.} $4\pi/5$
\quad \textbf{17.} $\pi/6$\\\textbf{19.} $-\pi/9$ \quad \textbf{21.} $12/13$ \quad
\textbf{23.} $\pi/2$ \quad \textbf{25.} $\pi/2$
\section*{Chapter 6}
\subsection*{Section 6.1 (p. \pageref{sec6dot1})}
\textbf{1.} $\frac{3\pi}{4} + \pi k$ \quad \textbf{3.} $\frac{3\pi}{10} + \frac{2\pi k}{5}$ \quad
\textbf{5.} $\pm\,\frac{\pi}{6} + \pi k$\\\textbf{7.} $-0.821 + 2\pi k$, $3.963 + 2\pi k$ \quad
\textbf{9.} $\frac{\pi}{4} + \pi k$\\\textbf{11.} $\frac{2\pi k}{3}$
\subsection*{Section 6.2 (p. \pageref{sec6dot2})}
\textbf{1.} $x=1.89549426703398093962$
\subsection*{Section 6.3 (p. \pageref{sec6dot3})}
\textbf{1.} $-1+i$ \quad \textbf{3.} $-13i$ \quad \textbf{5.} $-1-i$ \quad \textbf{7.} $i$\\
\textbf{9.} $-i$ \quad \textbf{11.} $i$ \quad \textbf{13.} $-i$ \quad \textbf{15.} $i$\\
\textbf{17.} Let $z=a+bi$. Then $\overline{z}=a-bi$, so $\overline{\left(\overline{z}\right)} =
\overline{a-bi}=a+bi=z$. \quad \textbf{23.} Hint: Use Exercise 20. \quad
\textbf{25.} $\sqrt{13}\,\text{cis}\;56.3\Degrees$ \quad
\textbf{27.} $\sqrt{2}\,\text{cis}\;315\Degrees$ \quad \textbf{29.} $\text{cis}\;0\Degrees$ \quad
\textbf{33.} $81\,\text{cis}\;56\Degrees$ \quad \textbf{35.} $1.5\,\text{cis}\;253\Degrees$ \quad
\textbf{37.} $\sqrt[6]{2}\,\text{cis}\;15\Degrees$, $\sqrt[6]{2}\,\text{cis}\;135\Degrees$,
$\sqrt[6]{2}\,\text{cis}\;255\Degrees$\\\textbf{39.} $\frac{1}{2} + \frac{\sqrt{3}}{2}\,i$,
$-1$, $\frac{1}{2} - \frac{\sqrt{3}}{2}\,i$ \quad \textbf{41.} $\text{cis}\;36\Degrees$,
$\text{cis}\;108\Degrees$, $\text{cis}\;180\Degrees$, $\text{cis}\;252\Degrees$,
$\text{cis}\;324\Degrees$
\subsection*{Section 6.4 (p. \pageref{sec6dot4})}
\textbf{1.} $(-3\sqrt{3},-3)$ \quad \textbf{3.} $(\sqrt{3},-1)$\\
\textbf{5.} $(-1/\sqrt{2},-1/\sqrt{2})$ \quad \textbf{7.} $(\sqrt{10},251.6\Degrees)$\\
\textbf{9.} $(2\sqrt{5},333.4\Degrees)$ \quad \textbf{11.} $r = 6\,\cos\;\theta$\\
\textbf{13.} $r^2 \,\cos\;2\theta = 1$ \quad \textbf{14.} $r = 3/(2 - \cos\;\theta)$
\end{multicols}
