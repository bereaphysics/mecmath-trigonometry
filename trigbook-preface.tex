\addchap{Preface}
This book covers elementary trigonometry. It is suitable for a one-semester course at the college
level, though it could also be used in high schools. The prerequisites are high school algebra and
geometry.

This book basically consists of my lecture notes from teaching trigonometry at Schoolcraft College
over several years, expanded with some exercises. There are exercises at the end of each section.
I have tried to include some
more challenging problems, with hints when I felt those were needed. An average student should be
able to do most of the exercises. Answers and hints to many of the odd-numbered and some of the
even-numbered exercises are provided in Appendix A.

This text probably has a more geometric feel to it than most current trigonometry texts.
That was, in fact, one of the reasons I wanted to write this book. I think that approaching the
subject with too much of an analytic emphasis is a bit confusing to students. It makes much of the
material appear unmotivated. This book starts with the ``old-fashioned'' right triangle approach to
the trigonometric functions, which is more intuitive for students to grasp.

In my experience, presenting the definitions of the trigonometric
functions and then immediately jumping into proving identities is too much of a detour from
geometry to analysis for most students.
So this book presents material in a very different order than most books today. For
example, after starting with the right triangle definitions and some applications, general (oblique)
triangles are presented. That seems like a more natural progression of topics, instead of leaving
general triangles until the end as is usually the case.

The goal of this book is a bit different, too. Instead of taking the (doomed) approach that students
have to be shown that trigonometry is ``relevant to their everyday lives'' (which inevitably comes
off as artificial), this book has a different mindset:
\emph{preparing students to use trigonometry as it is used in other courses}.
Virtually no students will ever in their ``everyday life'' figure out the height of a tree
with a protractor or determine the angular speed of a Ferris wheel.
Students are far more likely to need trigonometry in other courses (e.g. engineering, physics).
I think that math instructors have a duty to prepare students for that.

In Chapter 5 students are asked to use the free open-source software Gnuplot to graph
some functions. However, any program can be used for those exercises, as long as it produces
accurate graphs. Appendix B contains a brief tutorial on Gnuplot.

There are a few exercises that require the student to write his or her own computer program
to solve some numerical computation problems. There
are a few code samples in Chapter 6, written in the Java and Python programming languages, hopefully
sufficiently clear so that the reader can figure out what is being done even without knowing those
languages. Octave and Sage are also mentioned. This book probably discusses numerical issues more
than most texts at this level (e.g. the numerical instability of Heron's formula for the area of a
triangle, the secant method for solving trigonometric equations). Numerical methods probably should
have been emphasized even more in the text, since it is rare when even a moderately complicated
trigonometric equation can be solved with elementary methods, and since mathematical software is
so readily available.

I wanted to keep this book as brief as possible. Someone once joked that trigonometry is two weeks
of material spread out over a full semester, and I think that there is some truth to that.
However, some decisions had to be made on what material to leave out. I had planned to include
sections on vectors, spherical trigonometry - a subject which has basically vanished from
trigonometry texts in the last few decades (why?) - and a few other topics, but decided against it.
The hardest decision was to exclude Paul Rider's clever geometric proof of the Law of Tangents
without using any sum-to-product identities, though I do give a reference to it.

This book is released under the GNU Free Documentation License (GFDL), which allows others to not
only copy and distribute the book but also to modify it. For more details, see the included copy of
the GFDL. So that there is no ambiguity on this matter, anyone can make as many copies of this book
as desired and distribute it as desired, without needing my permission. The PDF version will always
be freely available to the public at no cost (go to \url{http://www.mecmath.net/trig}). Feel free to
contact me at \texttt{\href{mailto:mcorral@schoolcraft.edu}{mcorral@schoolcraft.edu}} for any
questions on this or any other matter involving the book (e.g. comments, suggestions, corrections,
etc). I welcome your input.

\begin{flushleft}
\emph{July 2009}\hspace{\stretch{1}}\textsc{Michael Corral}\\
\emph{Livonia, Michigan}
\end{flushleft}
